\chapter[Numerical Methods]{Numerical Methods for Fractional-Order Differential Equations}

Numerical methods are challenging for fractional-order systems because of the non-locality of the fractional derivative. 

\section{Direct Application of the Gr\"unwald-Letnikov Fractional Derivative}

Because the Gr\"unwald-Letnikov definition of the fractional derivative contains a limit, we can use it by taking the limiting term to be small as a computational approximation of the fractional derivative. The approach itself is fairly straight-forward, but numerical issues arise fairly quickly. We will illustrate both the approach and its limitations with an example.

Recall that the Gr\"unwald-Letnikov definition of the fractional derivative is
\begin{equation*}
  \frac{\d^\alpha f}{\d t^\alpha}(t) = \lim_{\Delta t \rightarrow 0} \frac{\sum_{k=0}^{\infty}\left(-1\right)^k \begin{pmatrix} \alpha \\ k \end{pmatrix} f \left( t - k \Delta t \right) }{\left( \Delta t \right)^\alpha}
\end{equation*}
where the binomial coefficient is generalized to non-integer arguments as
\begin{equation*}
  \begin{pmatrix} \alpha \\ k \end{pmatrix} = \frac{\Gamma \left( \alpha + 1 \right)}{\Gamma \left( k + 1 \right) \Gamma \left( \alpha - k + 1 \right)}.
\end{equation*}
In the case where all initial conditions of all orders are zero, the upper limit of the summation can be changed
\begin{equation*}
  \frac{\d^\alpha f}{\d t^\alpha}(t) = \lim_{\Delta t \rightarrow 0} \frac{\sum_{k=0}^{\lf \frac{t}{\Delta t} \rf}\left(-1\right)^k \begin{pmatrix} \alpha \\ k \end{pmatrix} f \left( t - k \Delta t \right) }{\left( \Delta t \right)^\alpha}.
\end{equation*}
This, of course, leads to the approximation
\begin{equation*}
  \frac{\d^\alpha f}{\d t^\alpha}(t) \approx \frac{\sum_{k=0}^{\lf \frac{t}{\Delta t} \rf}\left(-1\right)^k \begin{pmatrix} \alpha \\ k \end{pmatrix} f \left( t - k \Delta t \right) }{\left( \Delta t \right)^\alpha}, \qquad \Delta t \ll 1.
\end{equation*}

We will return to an example from the Chapter~\ref{ch:diffeq} to implement this.

\begin{example}
  Solve
  \begin{equation*}
    m \frac{\d^2 x}{\d t^2}(t) + q \frac{\d^\gamma x}{\d t^\gamma}(t) = f(t)
  \end{equation*}
  assuming all zero initial conditions. Because this is a numerical method, we need to choose numerical values for all the paramters, so let $m=q=1$ and $\gamma = 1/2$ and let $f(t)$ be a unit step input:
  \begin{equation*}
    \frac{\d^2 x}{\d t^2}(t) + \frac{\d^\frac{1}{2} x}{\d t^\frac{1}{2}} = 1.
  \end{equation*}

  The approach we will take is to convert this second-order differential equation into two first-order equations and use Euler's method to compute an approximate numerical solution. So, let $x_1 = x$ and $x_2 = \dot x$ so that
  \begin{equation*}
    \frac{\d}{\d t}\begin{bmatrix}
    x_1 \\ x_2
    \end{bmatrix} = 
    \begin{bmatrix}
      x_2 \\
    1 - \frac{\d^\frac{1}{2} x_1}{\d t^\frac{1}{2}}
    \end{bmatrix}
  \end{equation*}
  which gives the approximation
  \begin{equation*}
    \begin{bmatrix}
      x_1\left(t + \Delta t \right) \\
      x_2\left(t + \Delta t \right)
    \end{bmatrix} =
    \begin{bmatrix}
      x_2(t) \\
      1 - \sum_{k=0}^{\lf \frac{t}{\Delta t} \rf} \frac{(-1)^k  \begin{pmatrix} \gamma \\ k
      \end{pmatrix} f \left( t - k \Delta t \right)}{\left( \Delta t \right)^\gamma}
    \end{bmatrix}.
  \end{equation*}
\end{example}
