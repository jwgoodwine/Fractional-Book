\chapter{Fractional Derivative Definitions}

There are many definitions of fractional derivatives. In this chapter, we present a few of them along with their properties and compare and contrast them.

\section{Preliminaries}

From last chapter, it was obvious that one element to generalize from integer-order derivatives to allow for fractional or real-ordered derivatives, was the gamma function in cases where the only barrier to allowing a deriviative to take real values was a factorial. This section covers a couple other similar tools that we will need shortly.

\subsubsection{Cauchy's Formula for Repeated Integration}

It turns out we will more easily find a general formula for a fractional number of integrations, as opposed to differentiation. That is no problem, though, because, for example, if we want the $1/3$ derivative, we can integrate a function $2/3$ times and then compute the integer-order first derivative of the result, the law of indices (through the Fundamenal Theorem of Calculus) gives that the result what we want. 

\begin{theorem}
  Let $f(t)$ be continuous. Then the $n$th repeated integral of $f(t)$ is given by
  \begin{align}
   f^{(-n)}(t) &= \int_a^{t}  \ \int_a^{\sigma_1}  \int_a^{\sigma_2}  \int_a^{\sigma_3} \cdots \int_a^{\sigma_{n-1}} f(\sigma_n) d \sigma_n d \sigma_{n-1} \cdots d \sigma_1 \nonumber  \\
   &= \frac{1}{\left( n - 1 \right)!} \int_a^t \left( x - z \right)^{n-1} f(z) dz.
    \label{eq:cauchy}
  \end{align}
 \label{th:cauchy}
\end{theorem}
\begin{proof}
add the proof.
\end{proof}

This theorem should make some intuitive sense. If you had to evaluate the single integral, the way to to it would be to integrate by parts $n$ times to eliminate the $(x - z)$ term, which would give the muliple integral form of it.

If we ask how can we integrate a function a fractional number of times, though, it is similar to what was done in the first chapter. If we have
\begin{equation*}
  f^{(-n)}(t) = \frac{1}{\left( n - 1 \right)!} \int_a^t \left( t - z \right)^{n-1} f(z) dz
\end{equation*}
the only term containing the order of integration, $n$ where $n$ can not be a fraction is, again, the factorial. So we can just replace it with a gamma function 
\begin{equation}
  \boxed{ f^{(-\alpha)}(t) = \frac{1}{\Gamma \left( \alpha \right)} \int_a^t \left( t - z \right)^{\alpha-1} f(z) dz. }
  \label{eq:fracint}
\end{equation}
