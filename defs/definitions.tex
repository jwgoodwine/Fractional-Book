\chapter{Fractional Derivative Definitions}

There are many definitions of fractional derivatives. In this chapter, we present a few of them along with their properties and compare and contrast them.

\section{Preliminaries}

From last chapter, it was obvious that one element to generalize from integer-order derivatives to allow for fractional or real-ordered derivatives, was the gamma function in cases where the only barrier to allowing a derivative to take real values was a factorial. This section covers a couple other similar tools that we will need shortly.

\subsubsection{Cauchy's Formula for Repeated Integration}

It turns out we will more easily find a general formula for a fractional number of integrations, as opposed to differentiation. That is no problem, though, because, for example, if we want the $1/3$ derivative, we can integrate a function $2/3$ times and then compute the integer-order first derivative of the result, the law of indices (through the Fundamental Theorem of Calculus) gives that the result what we want. 

\begin{theorem}
  Let $f(t)$ be continuous. Then the $n$th repeated integral of $f(t)$ is given by
  \begin{align}
   f^{(-n)}(t) &= \int_a^{t}  \ \int_a^{\sigma_1}  \int_a^{\sigma_2}  \int_a^{\sigma_3} \cdots \int_a^{\sigma_{n-1}} f(\sigma_n) d \sigma_n d \sigma_{n-1} \cdots d \sigma_1 \nonumber  \\
   &= \frac{1}{\left( n - 1 \right)!} \int_a^t \left( x - z \right)^{n-1} f(z) dz.
    \label{eq:cauchy}
  \end{align}
 \label{th:cauchy}
\end{theorem}
\begin{proof}
add the proof.
\end{proof}

This theorem should make some intuitive sense. If you had to evaluate the single integral, the way to to it would be to integrate by parts $n$ times to eliminate the $(x - z)$ term, which would give the multiple integral form of it.

If we ask how can we integrate a function a fractional number of times, though, it is similar to what was done in the first chapter. If we have
\begin{equation*}
  f^{(-n)}(t) = \frac{1}{\left( n - 1 \right)!} \int_a^t \left( t - z \right)^{n-1} f(z) dz
\end{equation*}
the only term containing the order of integration, $n$ where $n$ can not be a fraction is, again, the factorial. So we can just replace it with a gamma function 
\begin{equation}
  \boxed{ f^{(-\alpha)}(t) = \frac{1}{\Gamma \left( \alpha \right)} \int_a^t \left( t - z \right)^{\alpha-1} f(z) dz. }
  \label{eq:fracint}
\end{equation}

\section{Summary of Important Functions}

In engineering there is a relatively limited collection of functions that are so useful that their properties become second nature. Fractional calculus adds to that collection, and this section presents some of them along with some of their most important properties.

\subsection{The Gamma Function}
The gamma function will appear just about everywhere where we deal with fractional derivatives. We have already seen an example. The integral representation of the gamma function is
\begin{equation}
  \boxed{ \Gamma(t) = \int_0^\infty e^{-z} z^{t-1} dz. }
  \label{eq:gammadef}
\end{equation}

In the case where $t$ is an integer, they way to compute the integral by hand would be to do so repeated by parts to work the exponent of $z$ in the integrand down to zero:
\begin{align*}
 \Gamma(t) &= 
  \int_0^\infty e^{-z} z^{t-1} dz \\ &= \left[ z^{t-1} \left(- e^{-z} \right) \right]^\infty_0 + \left(t-1 \right) \int_0^\infty e^{-z} z^{t-2} dz \\
&= 0 - 0 + \left( t - 1 \right) \Gamma(t-1).
\end{align*}
Comparing the last line to the right hand side of the line above it shows that analogous to $n \left(n - 1 \right)! = n!$, $\boxed{ \left( t - 1 \right) \Gamma(t - 1) = \Gamma(t) }$. Also, continuing to integrate by parts and knowing that the boundary terms will always continue to be zero, we have
\begin{align*}
  \Gamma(t) &= \left( t - 1 \right) \Gamma \left( t - 1 \right) \\
  &= \left[ \left( t - 1 \right) \left( t - 2 \right) \right] \Gamma \left( t - 2 \right) \\
 & \vdots \\
 &= \left[ \left(t - 1 \right) \left( t - 2 \right) \cdots 1 \right] \Gamma(1) \\
 &= \left( t - 1 \right)!,
\end{align*}
which proves that $\boxed{ \Gamma(t) = (t-1)!, t \in \mathbb Z }$, where $\mathbb Z$ is the set of natural numbers.

While the gamma function provides a nice generalization of the factorial for positive $t$, it is singular at zero and negative integer values as is illustrated in Figure~\ref{fig:gammaall}. This is a feature we will have to expect to see in fractional derivatives that use the gamma function. Singularities are usually considered ``bad things'' but they actually make some sense in this context as the following example illustrates.

\begin{figure}
  \centering
  \subimport{figs/}{gammaall}
  \caption{Gamma function for positive and negative real $t$ values.}
\label{fig:gammaall}
\end{figure}

\begin{example}
  Consider $f(t) = t$ and the fractional derivatives computed using Equation~\ref{eq:monomialfrac} that are illustrated in Figure~\ref{fig:singex}. Note that the singularity of the gamma function at $t=0$ can be seen as a way for the fractional derivatives between the zeroth and first derivatives to move between the two.
\end{example}

The value of the gamma function at some special values should be cataloged:
\begin{itemize}
  \item $\Gamma(1) = 1$
  \item $\Gamma\left(\frac{1}{2}\right) = \sqrt{\pi}$
  \item $\Gamma\left(\frac{3}{2}\right) = \frac{\sqrt{\pi}}{2}$
  \item $\Gamma\left(\frac{5}{2}\right) = \frac{3 \sqrt{\pi}}{4}$
\end{itemize}<++>

\begin{figure}
  \centering
  \subimport{figs/}{singex}
  \caption{Plot of $f(t) = t$ and its $0.1$, $0.9$ and first derivative.}
  \label{fig:singex}
\end{figure}
zr
\subsection{The Error Function and Complementary Error Functions}
These functions will be important as solutions to equations like
\begin{equation*}
  \frac{d^\frac{1}{2} x}{d t^\frac{1}{2}}(t) + x(t) = 1.
\end{equation*}

The error function is defined by
\begin{equation}
  \boxed{ \operatorname{erf}(t) = \frac{2}{\pi} \int_0^t e^{-u^2} du. }
  \label{eq:erf}
\end{equation}
Evaluating $\erf$ at specific values of $t$:
\begin{itemize}
  \item $\erf(0) = \frac{2}{\pi} \int_0^0 e^{-u^2} du = 0$.
  \item $\lim_{t \rightarrow \infty} \erf(t) = \frac{2}{\pi} \lim_{t \rightarrow\infty}\int_0^t e^{-u^2} du = 1$
  \item $\erf(-\infty) = -1$
\end{itemize}

\subsection{Mittag-Leffler Functions}

