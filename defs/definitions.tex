\chapter{Fractional Derivative Definitions}

There are many definitions of fractional derivatives. In this chapter, we present a few of them along with their properties and compare and contrast them.
\section{Summary of Important Functions}

In engineering there is a relatively limited collection of functions that are so useful that their properties become second nature. Fractional calculus adds to that collection, and this section presents some of them along with some of their most important properties. First we consider some basic computations.

\subsection{Preliminaries}

A few computations are needed before we study the functions. 

\subsubsection{Gaussian Integral}
The Gaussian integral is
\begin{equation}
  \boxed {\int_{-\infty}^\infty e^{-z^2} dz = \sqrt{\pi}, }
  \label{eq:gaussianintegral}
\end{equation}
and is the area under the curve in Figure~\ref{fig:gaussianintegral}. Clearly, also
\begin{equation*}
  \int_0^\infty e^{-z^2} dz = \frac{\sqrt{\pi}}{2}.
\end{equation*}

To see this, consider the square of the integral and switch to polar coordinates
\begin{align*}
  \left( \int_{-\infty}^\infty e^{-z^2} dz \right)^2 &= \left( \int_{-\infty}^\infty e^{-z_1^2} dz_1 \right) \left( \int_{-\infty}^\infty e^{-z_2^2} dz_2 \right) \\
  &= \int_{-\infty}^{\infty} \int_{-\infty}^{\infty} e^{-z_1^2 - z_2^2} dz_1 dz_2 \\
  &= \int_0^{2 \pi} \int_0^\infty r e^{-r^2} dr d\theta \\
  &= 2 \pi \int_0^\infty r e^{-r^2} dr \\
  &= \pi \int_{-\infty}^0 e^u du \qquad (u = -r^2) \\
  &= \pi,
\end{align*}
which shows
\begin{equation*}
  \int_{-\infty}^{\infty} e^{-z^2} dz = \sqrt{\pi}.
\end{equation*}

\begin{figure}
  \centering
  \subimport{figs/}{gaussianintegral}
  \caption{Gaussian integral.}
\label{fig:gaussianintegral}
\end{figure}


\subsubsection{Notation}
We will need notation for the ``floor'' and ``ceiling'' operations. For $\alpha \in \mathbb R$, $\lc \alpha \rc$ represents the smallest integer greater than $\alpha$. For example, $\lc 3.05 \rc = 4$. Correspondingly, $\lf \alpha \rf$ is the largest integer less than $\alpha$, so $\lf 4.99 \rf = 4$. 

Because there will be multiple definitions of a fractional derivative, we need a way to distinguish them. In general, $D^\alpha$ will stand for the derivative operator of order $\alpha$, and $D^{-\alpha}$ will represent integrating by a fractional number of times. As will be seen, because most definitions will involve integrals, the limits of integration will also impact the operation, so we need to have those in the definition, so $_tD^\alpha_0$ will be the derivative operator if $\alpha$ is positive and the $0$ and $t$ will be the limits of integration (described subsequently). 

\subsection{The Gamma Function}
The gamma function will appear just about everywhere where we deal with fractional derivatives. We have already seen an example. The integral representation of the gamma function is
\begin{equation}
  \boxed{ \Gamma(t) = \int_0^\infty e^{-z} z^{t-1} dz. }
  \label{eq:gammadef}
\end{equation}

In the case where $t$ is an integer, they way to compute the integral by hand would be to do so repeated by parts to work the exponent of $z$ in the integrand down to zero:
\begin{align*}
 \Gamma(t) &= 
  \int_0^\infty e^{-z} z^{t-1} dz \\ &= \left[ z^{t-1} \left(- e^{-z} \right) \right]^\infty_0 + \left(t-1 \right) \int_0^\infty e^{-z} z^{t-2} dz \\
&= 0 - 0 + \left( t - 1 \right) \Gamma(t-1).
\end{align*}
Comparing the last line to the right hand side of the line above it gives a recursion relation analogous to $n \left(n - 1 \right)! = n!$, 
\begin{equation}
  \boxed{ \Gamma(t) = \left( t - 1 \right) \Gamma(t - 1). }
  \label{eq:gammarecursion}
\end{equation}
Also, continuing to integrate by parts and knowing that the boundary terms will always continue to be zero, we have
\begin{align*}
  \Gamma(t) &= \left( t - 1 \right) \Gamma \left( t - 1 \right) \\
  &= \left[ \left( t - 1 \right) \left( t - 2 \right) \right] \Gamma \left( t - 2 \right) \\
 & \vdots \\
 &= \left[ \left(t - 1 \right) \left( t - 2 \right) \cdots 1 \right] \Gamma(1) \\
 &= \left( t - 1 \right)!,
\end{align*}
which proves that $\boxed{ \Gamma(t) = (t-1)!, t \in \mathbb Z }$, where $\mathbb Z$ is the set of natural numbers.

While the gamma function provides a nice generalization of the factorial for positive $t$, it is singular at zero and negative integer values as is illustrated in Figure~\ref{fig:gammaall}.\footnote{In fact, this integral definition we used in Equation~\ref{eq:gamma} is only valid for positive arguments. Other definitions, such as a series one, has to be used for zero and negative values for the definition to be complete and rigorous.} This is a feature we will have to expect to see in fractional derivatives that use the gamma function. Singularities are usually considered ``bad things'' but they actually make some sense in this context as the following example illustrates.

\begin{figure}
  \centering
  \subimport{figs/}{gammaall}
  \caption{Gamma function for positive and negative real $t$ values.}
\label{fig:gammaall}
\end{figure}

\begin{example}
  Consider $f(t) = t$ and the fractional derivatives computed using Equation~\ref{eq:monomialfrac} that are illustrated in Figure~\ref{fig:singex}. Note that the singularity of the gamma function at $t=0$ can be seen as a way for the fractional derivatives between the zeroth and first derivatives to move between the two.
\end{example}

\begin{figure}
  \centering
  \subimport{figs/}{singex}
  \caption{Plot of $f(t) = t$ and its $0.1$, $0.9$ and first derivative.}
  \label{fig:singex}
\end{figure}

The value of the gamma function at some special values should be cataloged.
\begin{itemize}
  \item $\boxed{ \Gamma(1) = 1. }$ This can be directly computed
    \begin{equation*}
      \Gamma(1) = \int_0^\infty e^{-z} z^{1 - 1} dz = \left. -e^{-z} \right|_0^\infty = 1.
    \end{equation*}
  \item $ \boxed{ \Gamma\left(\frac{1}{2}\right) = \sqrt{\pi}. }$ This can also be directly computed using the Gaussian integral from Equation~\ref{eq:gaussianintegral} 
    \begin{align*}
      \Gamma\left(\frac{1}{2}\right) &= \int_0^\infty e^{-z} z^{\frac{1}{2} - 1} dz 
      = \int_0^\infty e^{-z} z^{-\frac{1}{2}} dz \\ 
      &= 2 \int_0^\infty e^{-u^2} du \qquad (u^2 = z) \\
      &= \sqrt{\pi}.
    \end{align*}
  \item $ \boxed{ \Gamma\left(\frac{3}{2}\right) = \frac{\sqrt{\pi}}{2}. }$  This can be computed using Equation~\ref{eq:gammarecursion}
    \begin{equation*}
      \Gamma \left( \frac{3}{2} \right) = \left( \frac{3}{2} -1 \right) \Gamma \left( \frac{3}{2} - 1 \right) = \frac{1}{2} \sqrt{\pi}.
    \end{equation*}
  \item $\boxed{ \Gamma\left(\frac{5}{2}\right) = \frac{3 \sqrt{\pi}}{4}. }$ Similar to the previous one
    \begin{equation*}
      \Gamma \left( \frac{5}{2} \right) = \left( \frac{5}{2} - 1 \right) \Gamma \left( \frac{5}{2} - 1 \right) = \frac{3}{2} \frac{\sqrt{\pi}}{2} = \frac{3}{4} \sqrt{\pi}.
    \end{equation*}
  \item $\boxed { \Gamma \left( -\frac{1}{2} \right) = -2 \sqrt{\pi}.  }$  Similarly, for some negative values, this follows from the recursion relation in Equation~\ref{eq:gammarecursion}
    \begin{equation*}
      \Gamma \left( \frac{1}{2} \right) = \left( \frac{1}{2} - 1 \right) \Gamma \left( \frac{1}{2} - 1 \right) \qquad \Longleftrightarrow \qquad \Gamma \left( -\frac{1}{2} \right) = \frac{\Gamma \left( \frac{1}{2} \right)}{-\frac{1}{2}} = -2 \sqrt{\pi}.
    \end{equation*}
\end{itemize}

The gamma function also appears in fractional-order inverse Laplace transforms. Similarly to what we did previously, we will start with integer-order computations and then generalize. Consider the usual Laplace transform of $f(t) = t^n$ where $n \in \mathbb Z$
\begin{align*}
  \mathcal L \left\{ t^n \right\} &= \int_{0^-}^\infty t^n e^{-s t} dt \\
  &= \left. \left( - t^n \frac{1}{s} e^{-s t} \right) \right|_0^\infty + \frac{n}{s}  \int_{0^-}^\infty t^{n-1} e^{-st} dt \\
    &= \frac{n}{s}  \int_{0^-}^\infty t^{n-1} e^{-st} dt  \\
    & \vdots \\
    &= \frac{n!}{s^{n+1}} \int_{0^-}^\infty e^{-s t} dt  \\
    &= \frac{n!}{s^{n+1}}. 
\end{align*}

A perfectly reasonable, albeit not mathematically rigorous, inference at this point would be 
\begin{equation*}
  \boxed{ \mathcal L  \left\{ t^\alpha \right\} =  \frac{\Gamma \left( \alpha + 1 \right)}{s^{\alpha + 1}}. }
\end{equation*}
In fact, we can compute it directly
\begin{align*}
  \mathcal L \left\{ t^\alpha \right\} &= \int_{0^-}^\infty t^\alpha e^{-s t} dt \\
  &= \int_{0^-}^\infty \left( \frac{u}{s} \right)^\alpha \frac{e^{-u}}{s} du \qquad \left( u = st \right) \\
  &= \frac{1}{s^{\alpha + 1}} \int_{0^-}^\infty e^{-u} u^\alpha du \\
  &= \frac{\Gamma \left( \alpha + 1 \right)}{s^{\alpha + 1}}, \quad \alpha > -1.
\end{align*}

\subsection{The Error Function and Complementary Error Functions}
These functions will be important as solutions to equations like
\begin{equation*}
  \frac{d^\frac{1}{2} x}{d t^\frac{1}{2}}(t) + x(t) = 1.
\end{equation*}

The error function is defined by
\begin{equation}
  \boxed{ \erf(t) = \frac{2}{\sqrt{\pi}} \int_0^t e^{-z^2} dz. }
  \label{eq:erf}
\end{equation}
Note that it is like the Gaussian integral, but only over a subset of the range of the definite integral. The \emph{complementary error function} is the integral over the remaining part of the domain
\begin{equation}
  \boxed{ \erfc(t)  = \frac{2}{\sqrt{\pi}} \int_t^\infty e^{-z^2} dz. }
  \label{eq:erfc}
\end{equation}
Plots of both the error function and the complementary error function appear in Figure~\ref{fig:erferfc}. It is clear from the definitions and the plots that
\begin{equation}
  \erfc(t) = 1 - \erf(t).
  \label{eq:erfs}
\end{equation}

\begin{figure}
  \centering
  \subimport{figs/}{erferfc}
  \caption{The error function and complementary error function.}
  \label{fig:erferfc}
\end{figure}

Evaluating $\erf$ at specific values of $t$:
\begin{itemize}
  \item $ \boxed{ \erf(0) = \frac{2}{\sqrt{ \pi} } \int_0^0 e^{-u^2} du = 0. }$
  \item $ \boxed{ \erf(\infty) = \frac{2}{\sqrt{ \pi }} \int_0^\infty e^{-u^2} du = 1. }$
  \item $ \boxed{ \erf(-\infty) = -1. }$
\end{itemize}

\subsection{Mittag-Leffler Functions}
\emph{Mittag-Leffler Functions} are generalizations of the exponential function, and play a role in solutions to constant-coefficient, homogeneous linear fractional-order ordinary differential equations analogous to the exponential for integer order differential equations. As will be shown subsequently, just as $x(t) = c e^{-a t}$ is the solution to
\begin{equation*}
  \frac{dx}{dt}(t) + a x(t) = 0
\end{equation*}
the function $c t^{\alpha - 1} E_{\alpha,\alpha}\left(-a t^\alpha\right)$, where $E_{\alpha,\alpha}$ is the Mittag-Leffler function to be defined shortly, is the solution to 
\begin{equation*}
  \frac{d^\alpha x}{dt^\alpha}(t) + a x(t) = 0.
\end{equation*}

Recall the Taylor series of the exponential function about $t=0$ is
\begin{equation*}
  e^{t} = 1 + t + \frac{t^2}{2!} + \frac{t^3}{3!} + \cdots = \sum_{k=0}^\infty \frac{t^k}{k!}.
\end{equation*}

These days we can not help but replace factorials with gamma functions. However, just doing that in the previous equation does not generalize anything because
\begin{equation*}
  \sum_{k=0}^\infty \frac{t^k}{\Gamma \left( k + 1 \right)} = e^t,
\end{equation*}
and nothing is really changed. 

The \emph{one parameter} and \emph{two parameter} Mittag-Leffler functions put a coefficient in front of the $k$ and $1$ in the gamma function
\begin{equation}
  \boxed{ E_\alpha(t) = \sum_{k=1}^\infty \frac{t^k}{\Gamma \left( \alpha k + 1 \right)}, \quad \alpha > 0 }
  \label{eq:mlone}
\end{equation}
and
\begin{equation}
  \boxed{  E_{\alpha, \beta}(t) = \sum_{k=1}^\infty \frac{t^k}{\Gamma \left( \alpha k + \beta \right)}, \quad \alpha, \beta > 0. }
  \label{eq:mltwo}
\end{equation}

In order to gain some insight into these functions, let us see what the effect of varying the two parameters does. Figure~\ref{fig:mlfalpha} plots $E_{\alpha,t}(-t)$ for various values of $\alpha$. Observe that for negative values, smaller $\alpha$ values are ``stronger'' whereas for positive values of $t$ the opposite is basically the case. All of the curves go through the value of $1$ at $t=0$. The curves are more ``curved'' than the exponential for $\alpha$ values less than one, and less curved for $\alpha$ values greater than one.

\begin{figure}
  \centering
  \subimport{figs/}{mlfalpha}
  \caption{Mittag-Leffler functions, $E_{\alpha,1}(-t)$ for $\alpha = 0.25, 0.5, 0.75, 1, 1.25, 1.5, 1.75$ and $2$. Looking at the left part of the plot near $t = -1$, $\alpha = 0.25$ is the top curve, and they are in order down to $\alpha = 2$ for the bottom curve.}
  \label{fig:mlfalpha}
\end{figure}

Figure~\ref{fig:mlfbeta} illustrates $E_{1,\beta}(-t)$ for various $\beta$ values. The trend to observe is that around time $t=0$, the lowest curve corresponds to the smallest $\beta$ values, and each subsequent curve increased from that one correspond to increasing $\beta$ values. 

\begin{figure}
  \centering
  \subimport{figs/}{mlfbeta}
  \caption{Mittag-Leffler functions, $E_{1, \beta}(-t)$ for $\beta = 0.25, 0.5, 0.75, 1, 1.25, 1.5, 1.75$ and $2$. Near $t=0$ the lowest curve is for $\beta = 0.25$ and increased values in $\beta$ correspond to the curves above that in order.}
  \label{fig:mlfbeta}
\end{figure}

There are certain combinations of $\alpha$ and $\beta$ where $E_{\alpha,\beta}(t)$ is equal to a known function. Specifically
\begin{itemize}
  \item $\boxed{ E_{1,1}(t) = E_1(t) = e^t. }$
  \item $\boxed{ E_{\frac{1}{2},1} (t) = E_\frac{1}{2}(t) = x^{t^2} \erfc(-t)  }$
  \item $\boxed{ E_{1,2}(t) = \frac{e^t - 1}{t}  }$
  \item $\boxed{ E_{2,1}(t^2) \cosh(t) }$
  \item $\boxed{ E_{2,2 }(t^2) = \frac{\sinh(t)}{t}. }$
\end{itemize}


\section{Fractional Integration and Fractional Derivatives}

From last chapter, it was obvious that one element to generalize from integer-order derivatives to allow for fractional or real-ordered derivatives, was the gamma function in cases where the only barrier to allowing a derivative to take real values was a factorial. This section covers a couple other similar tools that we will need shortly.

\subsection{Fractional Integration}

It turns out we will more easily find a general formula for a fractional number of integrations, as opposed to differentiation. That is no problem, though, because, for example, if we want the $1/3$ derivative, we can integrate a function $2/3$ times and then compute the integer-order first derivative of the result, the law of indices (through the Fundamental Theorem of Calculus) gives that the result what we want. 

\begin{theorem}
  Let $f(t)$ be continuous. Then the $n$th repeated integral of $f(t)$ is given by
  \begin{align}
   f^{(-n)}(t) &= \int_a^{t}  \ \int_a^{\sigma_1}  \int_a^{\sigma_2}  \int_a^{\sigma_3} \cdots \int_a^{\sigma_{n-1}} f(\sigma_n) d \sigma_n d \sigma_{n-1} \cdots d \sigma_1 \nonumber  \\
   &= \frac{1}{\left( n - 1 \right)!} \int_a^t \left( x - z \right)^{n-1} f(z) dz.
    \label{eq:cauchy}
  \end{align}
 \label{th:cauchy}
\end{theorem}
\begin{proof}
add the proof.
\end{proof}

This theorem should make some intuitive sense. If you had to evaluate the single integral, the way to to it would be to integrate by parts $n$ times to eliminate the $(x - z)$ term, which would give the multiple integral form of it.

If we ask how can we integrate a function a fractional number of times, though, it is similar to what was done in the first chapter. If we have
\begin{equation*}
  f^{(-n)}(t) = \frac{1}{\left( n - 1 \right)!} \int_a^t \left( t - z \right)^{n-1} f(z) dz
\end{equation*}
the only term containing the order of integration, $n$ where $n$ can not be a fraction is, again, the factorial. So we can just replace it with a gamma function 
\begin{equation}
  \boxed{ f^{(-\alpha)}(t) = \frac{1}{\Gamma \left( \alpha \right)} \int_a^t \left( t - z \right)^{\alpha-1} f(z) dz. }
  \label{eq:fracint}
\end{equation}

\subsection{Fractional Derivative Definitions}

We will consider two basic approaches to generalizing the derivative to fractional orders. The first will use the fractional integration idea above along with the fundamental theorem of calculus. The second is an extension of the usual limit definition of the derivative that has a nice extension to the finite difference method in numerical methods.

At this point we have the ability to integrate a function by a fractional amount, \eg, $f^{\left( -1/2 \right)}$ is the one-half integral of $f$. The basic idea is that we can use the fact that integrals and derivatives are inverse oprations and, for example, integrate a function by, say, $2/3$ and then differentiate once to get the $1/3$ derivative. The difference between the first two definitions we will consider is simply whether we differentiate first and then fractionally integrate, or vice versa. That seemingly small difference actually has large consequences.

\subsubsection{Riemann-Liouville Fractional Derivative}

For the Riemann-Liouville derivative definition, we integrate first and then differentiate. $\sideset{^{RL}_0}{^\alpha_t}\D$

\subsubsection{Caputo Fractional Derivative}
$^C_0D^\alpha_t$

\subsubsection{Gr\"unwald-Letnikov Fractional Derivative}
